Funkcją tych nauk jest pomóc Tobie pogłębić Twoje zrozumienie rzeczywistości i zredukować samooszukiwanie się.
Ich celem nie jest sprawić, byś uwierzył w to, w co my wierzymy, 
ale aby zmotywować Cię do podjęcia się głębokiego i osobistego badania nad naturą Umysłu i Rzeczywistości 
-- najgłębszego z możliwych.
Twoja praca tutaj nie polega na przyjmowaniu naszych przekonań i wierzeń, ale na kontemplacji.
Uważaj, aby nie traktować kontemplacji jako dodatkowej, irytującej pracy, do której zmuszamy Cię kiedy wolisz robić coś innego.
Delektuj się możliwością poznania i doświadczenia swojej Świadomości.
W przeciwnym razie po prostu wpadniesz w przekonania i ideologię, jak każdy inny głupiec.

Nikt inny jak Ty nie zrozumie lepiej Twoich potrzeb i wizji, w której sie realizujesz.
Poznaj swoją własną prawdę, ona jest w Tobie.
Musisz być świadomy o naturze chorobliwości, aby móc się od niej uwolnić.
Ta praca nie jest perfekcyjnie czysta.
Jest zepsuta przez ego.
Znajdziesz nieprawdziwość w naszej pracy.
I jeśli nie będziesz ostrożny, pochłoniesz w sobie naszą korupcję.
Zastanów się, jak poważne to jest.
Nasza praca jest jedynie tak czysta, na ile my, jako kolektyw, moglibyśmy być w chwili, gdy ją tworzono.
Nie jesteśmy perfekcyjnie czyści, choć chcielibyśmy być bardziej.
Robimy, co możemy, biorąc pod uwagę, 
że sami indywidualnie musimy zadbać o własne przetrwanie i próbować nauczyć siebie samych kochać.
W każdych naukach znajdziesz poważne błędy i niedopatrzenia, oto jak trudna jest ta praca.
Możemy stale ją udoskonalać, ale nigdy nie osiągniemy perfekcji -- zostaje nam więc pokochać ten proces.
Żadnej pozycji nie można uznać za bezpieczną, a my nie jesteśmy wyjątkiem.
Bądź ostrożny i myśl za siebie.
Podejdź do tego odpowiedzialnie.

Nie interesuje nas tutaj ideologia duchowa czy filozoficzna.
To, co Cię ratuje, to kontemplacja i osobiste badanie, w które dobrowolnie i z radością się angażujesz.
To oczyszcza Cię z wszystkich ludzkich bzdur i złudzeń.
Stąd bierze się prawda.
Twoja prawda nie pochodzi od nas, ani od nikogo innego niż Ty.
Jako ranny uzdrowiciel, dotknięty miłością, zamierzasz użyć własnego umysłu, aby się oczyścić, 
tak jak kulturysta wykorzystuje lata podnoszenia ciężarów, aby wyrzeźbić swoje ciało.
Żadna ideologia nie dodaje mięśni do ciała, tylko trening.
Celem kulturystyki nie jest osiągnięcie jakiegoś końcowego rezultatu, 
ale czerpanie przyjemności z całego procesu podnoszenia ciężarów.
Więc zakochaj się w podnoszeniu.
Zakochaj się w kontemplacji!
Jeżeli bierzesz na poważnie swój rozwój i swoje życie, będziesz to robić przez kolejne paredziesiąt lat.
Nie da się osiągnąć osobistej doskonałości bez pokochania procesu.
Przestań ścigać się do jakichś wyimaginowanych wyników.
Zwolnij i delektuj się procesem.
To jest nauka płynąca z naszej sztuki.
Nie jesteśmy tutaj, aby uczyć Cię w co należy wierzyć.
Nie musisz w nic wierzyć, wystarczy że włożysz pracę w siebie samego przez cały ten czas, jaki Ci został.

Jeżeli w jakimkolwiek momencie studiowania naszej pracy znajdziesz części, 
które według Ciebie hamują którąkolwiek z funkcji naszych nauk,
więc ich roli w twoim samorozwoju,
chcemy abyś zignorował te części i podążał za własną prawdą.
Bądź szczery ze sobą, to pozwoli Tobie poznać własną wizję.
Twoja wizja mówi Ci, czym jesteś i do czego jesteś zdolny.
Każdy ma to w sobie, to jest Miłością.
Poznanie, doświadczenie i urzeczywistnienie tej wizji ma miejsce na domenie sztuki miłości.
To fundament filozofii Dotkniętych Miłością.
Cały ten akt jest poświęcony sztuce, która ma nauczyć Cię kochać i wyostrzyć narzędzia, 
którymi będziesz pracował przez całe swoje życie, codziennie.
Nie tylko jako artysta, ale przede wszystkim jako człowiek.

\ornamentbreak

Do rozwinięcia.
\vin Potencjalne tematy do poruszenia:

Miłość (rozwinięcie) \\
Świat jest prosty \\
Świadomość \\
Prawda \\
Uczucia i emocjonalna inteligencja \\
Epistemologia \\
Metafizyka \\
Akceptacja i wybaczenie \\
Sztuka i indywidualna ekspresja; ESTETYKA \\
Skupienie \\
Obserwacja \\
Szczęście \\
Wolność \\
Strach. Dlaczego boimy się prawdy? \\
Cierpienie \\
Dualizm \\
Mądrość i intuicja \\
Umysł, psychologia i neurobiologia (lub w akcie Nauka) \\
Ambicja i Wizja \\
Kreatywność i Destruktywność \\
Czas i Miejsce -- teraźniejszość, tu i teraz \\
Cierpliwość. Dlaczego wartościowe rzeczy wymagają rozwoju w czasie (lub w akcie Rozwój Osobisty) \\
Zmiana i zrozumieć nietrwałość \\
Śmierć
