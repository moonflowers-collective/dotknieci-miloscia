\section{Miłość}
\label{milosc}

% ==============================================================
\subsection{Czym jest Miłość?}
\label{milosc/milosc}

% ==============================================================
\subsection{Czym jest brak Miłości?}
\label{milosc/brak-milosci}

\emph{(notatka w uproszczeniu, będę do tego wracać by to rozwinąć)}
Brakiem/złamaniem Miłości są:
przemoc, znęcanie się, nękanie, nienawiść, złość, osąd, krytyka,
strach, kłamstwo, wyzysk, zdrada, oszustwo, kradzież,
demonizacja, moralizacja, obarczanie winą, nieakceptacja, 
brak szacunku, kpina, złamanie obietnic, projekcja, gaslighting, manipulacja,
obojętność wobec cierpienia, karanie - zwłaszcza za popełnione błędy,
zemsta, zamknięcie się w sobie, egoizm, narcyzm, samolubność,
nieświadomość, usiłowanie zmieniania kogoś wbrew jemu.

% ==============================================================
\subsection{Brutalność Miłości}
\label{milosc/brutalnosc-milosci}

% ==============================================================
\subsection{Praktyka Miłości}
\label{milosc/praktyka-milosci}

Kochać jest trudne. To najtrudniejsze, co można tu i teraz zrobić.
Odpowiemy sobie tutaj na fundamentalne pytanie, co oznacza kochać?

To niemożliwe, aby prawdziwie kochać wszystko, każdego człowieka, w zdrowy dla nas samych sposób.
Będąc niezdrowym w ten sposób, więc odbierając sobie samemu Miłości, nie będziemy w stanie prawdziwie kochać niczego.
Zrobisz sobie poważną krzywdę, jeśli spróbujesz praktykować w swoim życiu medytacje, którę Ci za chwilę przedstawię,
bez próby dogłębnego zrozumienia istoty tego, o czym mowa.
Zwłaszcza w niestabilnym stanie emocjonalnym, bądź ostrożny, aby nie zrobić nic głupiego.
To niemożliwe aby kochać, jeśli nie zaczniemy od siebie samych. 

Miłość wymaga poświęceń. 
Aby kochać, musimy wyzbyć się podziału, między Ja i Ty.
Wszyscy jesteśmy tym samym w doświadczeniu Miłości, bo Miłość jest rozpuszczeniem pozornego podziału między jednym, a drugim.
Jesteśmy jednością, z takim podejściem czytaj też kolejne medytacje na liście, którą Tobie przygotowaliśmy.
Dużo z przykładów będzie przyziemnych, poruszających naszą codzienność i bezpośrednio nas, ludzi.
Ale większość z zasad, które tutaj poznasz, mają zastosowanie nie tylko do innych, do rzeczy, ale przede wszystkim do Ciebie. Twoje Ja powinno być najważniejszym obiektem Twojej Miłości, od tego zależy wszystko inne.
Ty jesteś Miłością, dlatego tak ważne jest kochać.

Ta lista zasad tworzy postawę. 
Są to medytacje które powinno się praktykować, aby nauczyć siebie kochać w każdej chwili, tu i teraz.
Nie przejmuj się, jeśli coś nie zrozumiesz odrazu, albo jeśli nie trzymasz się całkowicie tej listy.
Nie chodzi o to, aby ślepo powielać tą praktykę, ani żeby karcić siebie lub innych za to, że nie postępują tak samo.
Potrzebne jest zrozumienie, akceptacja, skupienie oraz czas. Cierpliwość.
Wielkie zmiany nie przychodzą odrazu, są efektem drobnostek, które robimy codziennie.
A te drobnostki tworzą doskonałość.
To, co robisz tu i teraz, ma znaczenie. 
To zmienia świat wokół i Ciebie.

\ornamentbreak

\subsubsection{Poświęcaj Temu uwagę, uważność i czas.}
\emph{Do uzupełnienia.}

Wyobraź sobie, że próbujesz kochać kogoś, coś, cokolwiek, bez poświęcenia Temu swojego czasu i uwagi.
Czy bez tej świadomości, Miłość może w ogóle mieć miejsce?
Zastanów się, jak bardzo ważne są sam czas, uwaga oraz uważność w tym, aby kochać.
To poświęcenie tworzy głębokie więzi z obiektem twojej Miłości.

\subsubsection{Odczuwaj i pielęgnuj połączenie oraz więzi: fizyczne, emocjonalne, intelektualne i duchowe.}
Miłość to rozpuszczenie podziału między jednym i drugim, to bliskość, więzi, połączenie, jedność.
Myśl o praktyce Miłości, jako próbie zbudowania głębokiego połączenia z obiektem Twojej Miłości oraz zatarcia różnic i podziałów między wami.
Są różne rodzaje połączeń, jakie możesz budować: \emph{fizyczne, emocjonalne, intelektualne} oraz \emph{duchowe}.
Tak na prawdę nie chodzi o to, aby wybrać jeden rodzaj, ale aby pielęgnować je wszystkie w jakimś stopniu.
Jeżeli coś kochasz, będziesz starał się połączyć z tym w każdy dostępny dla Ciebie sposób.
Kochać siebie, to znaczy tworzyć połączenia z sobą samym, traktowanie świata i ludzi wokół jako rozszerzenie Ciebie samego.

\subsubsection{Bądź z Tym w fizycznej bliskości, w fizycznym kontakcie.}
Dotyk, trzymanie, przytulanie, tulenie, z czego najbardziej fundamentalna jest po prostu \textbf{bliskość}.
Już sama bliskość może zrodzić w nas głębokie uczucia, które utożsamiamy z miłością.
Pomyśl, gdy jesteś tu i teraz z tym, co próbujesz kochać -- jeżeli pozbawiasz was obu więzi i bliskości, jak drugie czuje się w tej chwili?

\subsubsection{Zadbaj o To, Tego potrzeby, przetrwanie i rozwój.}
Jeżeli coś kochasz, to oczywiste, że chcesz o to zadbać.
Zadbać o coś, to znaczy dbać o Tego potrzeby, niesienie pomocy Temu w dbaniu o siebie, pomaganie Temu przetrwać.
Bo jeśli próbujesz dbać o coś, to znaczy, że chcesz aby przetrwało.
Chcesz aby żyło i istniało.
Jeżeli coś kochasz, to znaczy kochasz to, że to coś istnieje tu i teraz.
Dostrzegasz piękno i wartość w istnieniu Tego, więc chcesz Temu pomóc istnieć dalej, gdy Tego istnienie jest zagrożone.

Problem w dbaniu o coś jest taki, że rodzi to przywiązanie i stronniczość, a to już może być niezdrowe.
Kochasz jakąś skończoną formę, którą masz przed sobą, lub którą ty sam jesteś. 
Zwłaszcza w Miłości do Ja trzeba uważać, by nie było to źródłem konformizmu i narcyzmu.
Jako ludzie przywiązujemy się do tego, co znane, więc do tego że obiekt Twojej Miłości istnieje teraz w sposób, jaki jest Ci znany.
Pomaganie czemuś przetrwać sprawia, że coś innego może zginąć albo się zmienić, to część brutalności jaka wiąże się z Miłością.
Dbanie o coś za wszelką cenę to na pewno stronnicza forma Miłości, ale nijak ma się to w porównaniu z krzywdą, jaką niesie ze sobą zaniedbanie.

Szanuj swoje ciało i umysł. Dbaj o zdrowe odżywianie, regularną aktywność fizyczną i praktykę medytacji.
Pozwól sobie odpocząć, brak odpoczynku i stres są winą zaniedbania.

\subsubsection{Przyjmij Tego zasady, priorytety i wizję, jako swoje własne.}
Poważny błąd, jaki popełniamy próbując kogoś kochać, to narzucamy mu swoją własną wizję, priorytety i zasady.
Te nazywa się czyimś porządkiem dziennym, bo chociaż mowa o czymś większym od nas samych, te mówią nam wprost o działaniach, którymi sie je praktykuje i realizuje codziennie.
Błąd polega na przyjęciu narcystycznej implikacji, że skoro kochamy kogoś, to musi być taki jak My -- więc podziela nasz własny porządek dzienny.
Przykładem mogą być rodzice, np.\ gdy dziecko oznajmi, że w przyszłości chciałoby być artystą, spotkałby się z odzewem najbliższych w stylu: \emph{-- Nie, zostaniesz prawnikiem, tak jak twój ojciec. -- Nie, nie ma miejsca na sztukę w tym domu.}
Taka postawa jest krzywdząca, jest podstawą do wzniecania buntu i konfliktów, do budowania nienawiści między sobą, do zrywania więzi i zaniedbania.
Więc, jakbyś traktował swoje dziecko, gdybyś miał traktować je jak siebie samego?

\subsubsection{Przyjmij Tego wizję, jako rozszerzenie Twojej -- jako rozszerzenie Ciebie samego.}
Zamiast robić z kogoś rozszerzenie Twojej wizji, przyjmij jego wizję jako rozszerzenie Ciebie samego.
Literalnie co robisz, gdy kogoś kochasz, to bierzesz swoją skończoną tożsamość i rozszerzasz ją o obiekt swojej Miłości.
To trudne i wręcz straszne do zrobienia, bo to buduje wrażliwość i otwiera Cię na słabości, co inni mogą wykorzystać przeciwko Tobie.
Gdy rozszerzasz swoją wizję, swoją tożsamość o coś lub kogoś, tworzysz w ten sposób głębokie połączenie.
Gdy temu stanie się krzywda, tobie również staje się krzywda.
Jesteś związany z przetrwaniem Tego, z bólem który odczuwa, ale też bezpośrednio wiążesz się z Miłością, która płynie od Tego.

\subsubsection{Traktuj innych tak samo dobrze, jak chciałbyś traktować samego siebie.}
a 
\subsubsection{Szczerze ciesz się, gdy Temu się w czymś powiedzie.}
a 
\subsubsection{Chciej dla kogoś tego, co chce dla siebie samego, nawet jeśli jest to inne od tego, czego ty byś chciał dla siebie.}
a 
\subsubsection{Poświęcaj się i pracuj na rzecz Tego.}
a 
\subsubsection{Wspieraj, pielęgnuj i zachęcaj w tym, co chce dla siebie samego.}
a 
\subsubsection{Respektuj Tego suwerenność.}
a 
\subsubsection{Całkowicie akceptuj To takim, jakim jest. Nie usiłuj Tego zmieniać wbrew niemu.}
Przyjmij siebie samego takim, jakim jesteś.
Akceptuj swoje wady i niedoskonałości, to jest piękne, to sprawia że jesteśmy ludźmi.
\subsubsection{Nie potrzebuj niczego od Tego innego.}
a 
\subsubsection{Respektuj Tego światopogląd i chciej go zrozumieć.}
a
\subsubsection{Słuchaj, co próbuje się Tobie mówić.}
a 
\subsubsection{Troszcz się o Tego interesy oraz współdziel z nim swoje własne.}
a 
\subsubsection{Praw werbalne aprobaty, komplementy i pochwały.}
a 
\subsubsection{Bądź hojny, życzliwy i dobry.}
Oferuj pomoc i wsparcie, gdy to potrzebne.
\subsubsection{Broń Tego, bądź Temu lojalny.}
Jesteś w stanie powiedzieć, gdy ktoś jest Tobie lojalny.
Albo gdy ktoś nie jest -- jest Twoim wrogiem, knuje przeciwko Tobie za Twoimi plecami, darzy Cię skrytą lub otwartą nienawiścią.
To można wyczuć swoją intuicją.
Zdajesz sobie sprawę, że nie można zaufać ludziom, którzy decydują się Ciebie zaniedbać lub stają przeciwko tobie.
Nie można na nich polegać, bo nie są Tobie lojalni.
Obcowanie z ludźmi, których nie darzysz zaufaniem, buduje w Tobie stres i wyizolowanie.
Nie pozwól na siebie wpływać ludziom, którzy są przeciwko Tobie i Twoim interesom.
Bądź z ludźmi, którzy są dla Ciebie i darzą Cię Miłością.
Lojalność i zaufanie raz złamane praktycznie niemożliwym jest odbudować.
Dlatego pamiętaj o tym, aby nigdy nie pozwolić sobie tego złamać.
Broń ludzi, których kochasz.
Bądź lojalny dla nich i stań po ich stronie.

\subsubsection{Buduj oraz utrzymuj zaufanie i poczucie bezpieczeństwa.}
a 
\subsubsection{Buduj poczucie wspólnoty, związku, przyjaźni i współpracy.}
a 
\subsubsection{Uchroń To przed nadmierną traumą, strachem i cierpieniem, które może To wyniszczyć i pokaleczyć.}
To jest tak na prawdę rola, jaką pełnią rodzice, że próbują uchronić swoje dziecko przed traumą i cierpieniem.
Oczywiście, nie można dopuścić, by dziecko nigdy nie doświadczyło strachu ani cierpienia.
Nadopiekuńczość skrzywdzi je tak samo, a nawet bardziej, jak nadmierne cierpienie, gdy to dziecko wejdzie w dorosłość i zacznie być odpowiedzialne za siebie.
Nie uciekaj od cierpienia i chorobliwości, bo od tego nie uciekniesz.
To ważne, aby nauczać jak cierpieć świadomie oraz by nie trzymać tego cierpienia w sobie i pozwolić mu odejść.
Wszyscy cierpimy wystarczająco, niepotrzebnie.
Nakładając sobie więcej niepotrzebnego stresu i cierpienia na siebie dążymy do samodestrukcji.
Może to nie być oczywiste, ale takie chorobliwe drobnostki budujące się na nas mogą nas poważnie skrzywdzić, a nawet zabić.

W kontekście nadopiekuńczości -- jeżeli coś kochasz, nie chcesz aby to było uzależnione od Ciebie do końca swojego życia.
Chcesz aby stało się silniejsze i w końcu niezależne od Ciebie.
A wręcz chcesz, aby stało się silniejsze od Ciebie i rozwinęło się bardziej niż Ty, dzięki Tobie.
W przypadku Miłości do Ja mamy podobną sytuację, tylko porównujemy siebie tu i teraz do swojego Ja z wczoraj.
Chcemy być silniejsi, niż byliśmy wcześniej. 
Chcemy się stale rozwijać i rosnąć, przekraczać własne limity.
To rola Miłości w tym, aby rosnąć w zdrowy dla nas samych sposób.

W kontekście zasad zdrowej komunikacji -- ludzie są i będą niezdrowi, nieświadomi, chorobliwi, egoistyczni.
Jeżeli nie potrafisz w zdrowy sposób podejść do niezdrowego człowieka, to znaczy nie jesteś gotowy kochać Tego, co Ciebie nienawidzi, kogoś tak nienawistnego i nieświadomego, to Twoja odpowiedzialność, aby siebie samego przed tym uchronić.
Nie angażuj się w niepotrzebne konflikty.
Unikaj relacji i ludzi, którzy wciągają Cię w swoją chorobliwość.
Uchroń siebie przed niepotrzebnym cierpieniem.

\subsubsection{Bądź wdzięczny za to, co masz.}
Zamiast skupiać się na tym, co Ci brakuje.
\subsubsection{Bądź otwarty na naukę i nowe doświadczenia.}
Zwłaszcza jeśli wymagają od Ciebie wyjścia poza Twoją strefę komfortu.
Odkrywaj siebie samego. Poznaj swoje pasje i wizję, której możesz się poświęcić.
\subsubsection{Bądź dla Tego kogoś, gdy czuje się gorzej, żałośnie, słabo, beznadziejnie, zdesperowany, biednie.}
a 
\subsubsection{Pomagaj Temu nie czuć się samotnie.}
a 
\subsubsection{Zatwierdzaj, współdziel i dbaj o uczucia Tego. Nie ignoruj ani nie zaniedbuj uczuć.}
a 
\subsubsection{Troszcz się o Tego cierpienie, dla cierpienia samego w sobie.}
a 
\subsubsection{Utrzymuj pokój. Nie wprowadzaj niepotrzebnych konfliktów.}
a 
\subsubsection{Bądź z Tym na równych warunkach.}
a 
\subsubsection{Okazuj Temu zrozumienie i wybaczenie.}
a 
\subsubsection{Bądź cierpliwy i zaczekaj.}
a 
\subsubsection{Nie uciekaj, nie usiłuj, ani nie bądź bierny.}
a 
\subsubsection{Dostrzegaj Miłość w Tym, nawet jeśli ono samo w sobie tej Miłości nie dostrzega.}
a 
\subsubsection{Rozpoznaj w Tym niepowtarzalność i unikatowość.}
a 
\subsubsection{Dostrzegaj w tym wewnętrzne, nieskończone piękno.}
a 
\subsubsection{Dotrzymuj swoich obietnic.}
a 
\subsubsection{Mów prawdę i bądź prawdziwy.}
a 
\subsubsection{Dostrzegaj w Tym jego prawdziwość.}
a 
\subsubsection{Całkowicie akceptuj i nie bądź niezrównoważonym przez Tego egoizm, ani nie usiłuj Tego postawy zmieniać.}
a 
\subsubsection{Pozwól odejść Temu, co kochasz. Nie usiłuj Tego zatrzymać dla siebie wbrew Temu.}
a 
\subsubsection{Podaruj całego siebie Temu, co kochasz.}
a 

% ==============================================================
\subsection{Jak zakochać się w życiu?}
\label{milosc/jak-zakochac-sie-w-zyciu}

% ==============================================================
\subsection{Miłość do siebie}
\label{milosc/milosc-do-siebie}
