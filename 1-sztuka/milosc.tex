\documentclass[../dotknieci-miloscia.tex]{subfiles}

\begin{document}


\section{Miłość}
\label{sztuka/milosc}
Rozdział jest zdecydowanie niekompletny. Część „Praktyka Miłości” co prawda jest relatywnie obszerna, 
ale wciąż wymaga dalszej pracy i redakcji.

Będę pracować nad tym rozdziałem w kolejnym miesiącu.
\quad \textsc{-- violacoda}

\ornamentbreak
\vspace*{-1em}
\begin{multicols}{2}


\subsection{Natura Miłości}
\label{sztuka/milosc/definicja}

Dopóki ta część pozostaje nierozwinięta, 
zostawię tutaj myśli dotyczące natury Miłości i tego, co oznacza kochać.

\emph{Satchit ananda} : \emph{sat} -- Prawda; \emph{chit} -- Świadomość; \emph{ananda} -- Miłość.

Będziemy mówić tutaj o Miłości jako synonimie Prawdy i Świadomości, 
obdartej z kulturowej otoczki i romantyczności, jaką z nią wiążemy.

Duchowość jest sztuką miłości.

Z metafizyki, Miłość jest właściwością wszechświata.
Miłość jest wszystkim.

Miłość jest rozpuszczeniem pozornego podziału między jednym, a drugim.

Nie czekaj aż Miłość zapuka Ci do drzwi.

Uświadomienie sobie, że każdy osąd jest złudzeniem.

Praktykuj uczucie miłości, zwłaszcza w chwilach gdy czujesz się najgorzej i najsłabiej.
To jest istota praktyki medytacji.

Miłość wymaga poświęceń. 

To niemożliwe, aby prawdziwie kochać wszystko, każdego człowieka, w zdrowy dla nas samych sposób.
Odbierając sobie samemu Miłości, nie będziemy w stanie prawdziwie kochać niczego.

To niemożliwe aby kochać, jeśli nie zaczniemy od siebie samych. 

Zrobisz sobie poważną krzywdę, jeśli spróbujesz praktykować w swoim życiu Miłość,
bez próby dogłębnego zrozumienia istoty tego, czym jest i jaka jest jej natura.
Zwłaszcza w niestabilnym stanie emocjonalnym, bądź ostrożny, aby nie zrobić nic głupiego.

Ego wie jedynie jak kochać warunkowo do tego, co wspiera jego przetrwanie i stymulację.

Osoba, która nienawidzi, jest niezdolna kochać samego siebie.

Narcyzm nie jest Miłością.
Narcyz kocha w sobie to, co jest łatwe pokochać,
a zaprzecza i tłumi inne części siebie, dokonując projekcji tego na świat wokół.

Twoja kariera i związki, jakie budujesz, powinny być ekspresją Twojej miłości do świata.
W innym wypadku, po co to robisz?
Tracisz jedynie czas.


\subsection{Brutalność Miłości}
\label{sztuka/milosc/brutalnosc}

Nienawiść nie jest przeciwieństwem miłości.
Nienawiść jest przesadną miłością do Ja, która skutkuje brakiem miłości do innego.
To bardzo gęsta i prymitywna forma miłości.
To zniekształcona, źle skierowana miłość.

Każde zło dzieje się w dobrych intencjach.
Każda zła osoba czyni z Miłości.
Ponieważ co ma miejsce, to ego stworzyło podział między sobą a całą resztą świata.
Ego boi się poddać i zatrzeć ten podział, bo od tego zależy jego przetrwanie.
Co kieruje złem to nieświadomość, strach przed tym, aby stracić ego.

Prawdziwym sprawdzianem w miłości jest umiejętność do kochania wszystkiego, 
nie tylko rzeczy, które zadowalają Twoje ego.

Konfucjusz: „If you hate a person, you're defeated by them."

Mahatma Gandhi: „Real love is to love them that hate you.”

Nie ma gwarancji, że odznaczając się postawą Miłości, wszystko pójdzie po Twojej myśli.


\end{multicols}
\vspace*{-1.5em}
\ornamentbreak


\subsection{Praktyka Miłości}
\label{sztuka/milosc/praktyka}

Celem tej części rozprawy jest wyczerpująco poruszyć temat praktyki Miłości.
Odpowiemy sobie na fundamentalne pytanie, \emph{co oznacza kochać?}
Jaką postawą i jakimi czynami odznacza się człowiek, który coś lub kogoś kocha?
Poniżej znajdziesz listę parudziesięciu medytacji, nad którymi będziemy kontemplować.
Spróbuj samemu zastanowić się nad każdym punktem i co on oznacza dla Ciebie, 
możesz podzielić się tym z nami na forum.

Upewnij się, że kochasz nie dlatego, że powinieneś, że musisz, albo że jest taka zasada.
Nieszczera praktyka nie jest miłością.

Kochasz, ponieważ kochasz. 
To jest czym jesteś.

\subsubsection{Poświęcaj temu uwagę, uważność i czas.}
Wyobraź sobie, że próbujesz kochać kogoś, coś, cokolwiek, 
bez poświęcenia Temu swojego czasu i uwagi.
Czy bez tej świadomości, miłość może w ogóle mieć miejsce?
Zastanów się, jak bardzo ważne są sam czas, uwaga oraz uważność w tym, aby kochać.
To poświęcenie tworzy głębokie więzi z obiektem Twojej miłości.
\emph{Do rozwinięcia.}

\subsubsection{Odczuwaj i pielęgnuj połączenie oraz więzi: fizyczne, emocjonalne, intelektualne i duchowe.}
Miłość to rozpuszczenie podziału między jednym i drugim, to bliskość, więzi, połączenie, jedność.
Myśl o praktyce miłości, jako próbie zbudowania głębokiego połączenia oraz zatarcia różnic i podziałów między wami.
Są różne rodzaje połączeń, jakie możesz budować: 
\emph{fizyczne, emocjonalne, intelektualne} oraz \emph{duchowe}.
Tak na prawdę nie chodzi o to, aby wybrać jeden rodzaj, ale aby pielęgnować je wszystkie.
Jeżeli coś kochasz, będziesz starał się połączyć z tym w każdy możliwy dla Ciebie sposób.
Kochać siebie, to znaczy tworzyć połączenia z sobą samym, 
zatarcie różnic między Tobą, a światem, 
traktowanie świata i ludzi wokół jako rozszerzenie Ciebie samego.

\subsubsection{Bądź z tym w fizycznej bliskości, w fizycznym kontakcie.}
To są np.\ dotyk, trzymanie, przytulanie, tulenie, 
z czego najbardziej fundamentalna jest po prostu \textbf{bliskość}.
Już sama bliskość może zrodzić w nas głębokie uczucia, które utożsamiamy z miłością.
Pomyśl, gdy jesteś tu i teraz z tym, co próbujesz kochać 
-- jeżeli pozbawiasz was obu więzi i bliskości, jak drugie czuje się w tej chwili?
\emph{Do rozwinięcia.}

\subsubsection{Zadbaj o to, tego potrzeby, przetrwanie i rozwój.}
Jeżeli coś kochasz, to oczywiste, że chcesz o to zadbać.
Zadbać o coś, to znaczy dbać o Tego potrzeby, 
niesienie pomocy Temu w dbaniu o siebie, pomaganie Temu przetrwać.
Bo jeśli próbujesz dbać o coś, to znaczy, że chcesz aby przetrwało.
Chcesz aby żyło i istniało.
Jeżeli coś kochasz, to znaczy kochasz to, że to coś istnieje tu i teraz.
Dostrzegasz piękno i wartość w istnieniu tego, 
więc chcesz temu pomóc istnieć dalej, gdy Tego istnienie jest zagrożone.

Problem w dbaniu o coś jest taki, że rodzi to przywiązanie i stronniczość, a to już może być niezdrowe.
Kochasz jakąś skończoną formę, którą masz przed sobą, lub którą ty sam jesteś. 
Zwłaszcza w miłości do Ja trzeba uważać, by nie było to źródłem konformizmu i narcyzmu.
Jako ludzie przywiązujemy się do tego, co znane, 
więc do tego że to, co kochasz, istnieje teraz w sposób, jaki jest Ci znany.
Pomaganie czemuś przetrwać sprawia, że coś innego może zginąć albo się zmienić, 
to część brutalności jaka wiąże się z Miłością.
Dbanie o coś za wszelką cenę to na pewno stronnicza forma Miłości, 
ale nijak ma się to w porównaniu z krzywdą, jaką niesie ze sobą zaniedbanie.

Szanuj swoje ciało i umysł. Dbaj o zdrowe odżywianie, regularną aktywność fizyczną i praktykę medytacji.
Pozwól sobie odpocząć, brak odpoczynku i stres skutkują zaniedbaniem.

\subsubsection{Przyjmij tego porządek dzienny: zasady, priorytety i wizję, jako swoje własne.}
Poważny błąd, jaki popełniamy próbując kogoś kochać, 
to narzucamy mu swoją własną wizję, priorytety i zasady.
Te nazywa się czyimś porządkiem dziennym, bo chociaż mowa o czymś większym od nas samych, 
te mówią nam wprost o działaniach, którymi sie je praktykuje i realizuje codziennie.
Błąd polega na przyjęciu narcystycznej implikacji, że skoro kochamy kogoś, to musi być taki jak My 
-- jest to projekcja naszej wizji, zasad i priorytetów, na kogoś.

Przykładem mogą być rodzice, np.\ gdy dziecko oznajmi, że w przyszłości chciałoby być artystą, 
spotkałby się z odzewem najbliższych w stylu: 
\emph{
    -- Nie, zostaniesz prawnikiem, tak jak twój ojciec. 
    -- Nie, nie ma miejsca na sztukę w tym domu.}
Taka postawa jest krzywdząca, jest podstawą do wzniecania buntu i konfliktów, 
do budowania nienawiści między sobą, do zrywania więzi i zaniedbania.
Jakbyś traktował swoje dziecko, gdybyś miał traktować je jak siebie samego?

\subsubsection{Przyjmij tego wizję, jako rozszerzenie Twojej -- jako rozszerzenie Ciebie samego.}
Zamiast robić z kogoś rozszerzenie Twojej wizji, 
przyjmij jego wizję jako rozszerzenie Ciebie samego.
Literalnie co robisz, gdy kogoś kochasz, 
to bierzesz swoją skończoną tożsamość i rozszerzasz ją o to, co kochasz.
To trudne i wręcz straszne do zrobienia, bo to buduje wrażliwość i otwiera Cię na słabości, 
co inni mogą wykorzystać przeciwko Tobie, gdy będziesz nieostrożny.
Gdy rozszerzasz swoją wizję, swoją tożsamość o coś lub kogoś, 
wrażliwością tworzysz głębokie połączenie.
Gdy temu stanie się krzywda, tobie również staje się krzywda.
Jesteś związany z przetrwaniem tego, z bólem który odczuwa, 
ale też bezpośrednio wiążesz się z miłością, która od tego płynie.

\subsubsection{Traktuj innych tak samo dobrze, jak chciałbyś traktować samego siebie.}
Wynika bezpośrednio z tego, co sobie powiedzieliśmy o miłości. 
Miłość to zatarcie pozornych różnic między jednym, a drugim.
„Jeśli jestem niemiły i okrutny dla siebie, dzielę tę postawę z Tobą.”
\emph{Do rozwinięcia.}

\subsubsection{Szczerze ciesz się, gdy temu się w czymś powiedzie.}
Czyiś sukces to twój sukces. 
Celebruj ten sukces. 
Ciesz się z kimś lub za kogoś w tym, że mu się powodzi i jest u niego lepiej. 
Jeżeli traktujesz innych na równi z Tobą, jako rozszerzenie Ciebie, 
cieszysz się z ich sukcesu tak samo, jak cieszysz się z własnego sukcesu. 
Sytuacja odwrotna, kiedy czujemy zazdrość, kiedy kwestionujemy ten sukces, 
kiedy zadajemy sobie pytania "Dlaczego on, oni, a nie Ja?" 
-- tworzysz podziały między Ja i innymi. 
Buduj empatię z ludźmi, nie tylko w żałobie i cierpieniu, ale też w sukcesie.

\subsubsection{Chciej dla kogoś tego, co chce dla siebie samego, nawet jeśli jest to inne od tego, czego ty byś chciał dla siebie.}
Błędne jest przyjęcie, że gdy mamy jakieś cele i wartości dla siebie, 
w czym się spełniamy, to zakładamy że inni mają takie same jak my.
Porusza to ten sam problem, co o porządku dziennym, ale dotyka trochę innego podejścia.
To nie zawsze jest prawdą, że inni podzielają nasz światopogląd.
W końcu mamy inne wizje dla siebie samych, inne potrzeby jakie musimy spełnić. 
Więc chodzi tutaj o to, by odkryć czyjąś wizję dla siebie i go w tym wspierać, 
nawet jeśli ta wizja jest inna od twojej i nawet jeśli osobiście się z nią nie zgadzasz 
(znowu przykład z rodzicami, „Nie ma miejsca na sztukę w tym domu”). 
Błędem byłoby tutaj przekonywanie kogoś, 
że to czego chce dla siebie samego, jeśli zdrowe dla niego, jest niepoprawne. 
Ale jeśli prawdziwie kogoś kochamy, 
to chcielibyśmy tego dla nich, bo oni tego chcą dla siebie samych. 

\subsubsection{Wspieraj, pielęgnuj i zachęcaj w tym, co chce dla siebie samego.}
Gdy kogoś kochasz, to będziesz go zachęcać do tego, aby być najlepszą wersją siebie. 
Aby ktoś spełnił się w swojej wizji, na swój unikatowy sposób. 
Nie dlatego, że potrzebujesz aby ten ktoś był lepszy dla siebie 
-- dla twoich potrzeb i pragnień 
-- ale dlatego, że wspierasz jego rozwój osobisty, samospełnienie i indywidualną ekspresję. 
Twoje zachęcanie i wspieranie kogoś musi być równoległe z tym, 
co ten ktoś chce dla siebie, w czym się pasjonuje. 
Chodzi o to, aby wspierać ludzi, których kochasz, 
w ich własnych celach i potrzebach, kreatywnych, duchowych, fizycznych, intelektualnych, etc.

\subsubsection{Poświęcaj się i pracuj na rzecz tego.}
Jeżeli prawdziwie kogoś kochasz, jesteś gotów poświęcić się dla tego i pracować w tego interesie. 
Czy to jest Twoja wizja, pasja, druga osoba, nieważne. 
Znowu przykład: w społecznościach jak redpill, incel itd., 
taką toksyczną męskość łączy tania mentalność: 
„Kobiety powinny pracować dla siebie samych, ja nie powinienem pracować w jej interesie, 
dbać o nią, rozpieszczać ją, ona sama powinna to robić”. 
Oczywiście, stojąc po stronie tych mężczyzn nie chcesz, 
aby kobieta wykorzystywała Cię dla Twoich pieniędzy. 
Ale jeśli na prawdę tworzysz głęboką relację z kobietą, którą kochasz, 
powinieneś pracować w jej interesie, musisz być gotowy poświęcać się dla niej. 
I to powinno działać w obie strony, 
jeśli ona Ciebie kocha, ona powinna być gotowa poświęcić się dla Ciebie. 
Oczywiście, mężczyzna i kobieta robią to w swój unikatowy dla siebie sposób, 
ale rzecz o związkach i męskości/kobiecości potrzebowałoby własnego tematu do rozwinięcia.
Trzeba być ostrożnym tutaj. 
Bardzo łatwo wpakować siebie samego w dysfunkcyjny związek, 
uzależnić się od kogoś i zbudować relacje na niezdrowym fundamencie. 
Bardzo łatwo dać się wykorzystać lub odwrotnie, wykorzystać kogoś. 
Rolą rozwiązania tego mają zająć się zasady zdrowej komunikacji, 
więc można to rozwinąć przy okazji tego rozdziału, gdy zacznie powstawać.

\subsubsection{Respektuj tego suwerenność.}
Jeżeli nie respektujesz czyjejś suwerenności, to nie możesz tego kogoś kochać. 
Respektować czyjąś suwerenność, to znaczy traktujesz kogoś jako ktoś równy i niezależny. 
Nie uznajesz siebie za kogoś lepszego albo dominującego, 
ani za kogoś gorszego i podwładnego. 
Nie manipuluj tą osobą. 
Nie próbuj ją kontrolować. 
Nie próbuj jej wykorzystać. 
Nie próbuj uznawać siebie za ważniejszego od kogoś, 
zdusić go swoją pozycją nad nim w społeczności. 
Bycie liderem, więc świadome przywództwo, 
wymaga aby uznać suwerenność każdego w społeczności
i nie uznawać siebie za kogoś wyższego od ludzi, których się prowadzi. 
Respektowanie czyjejś suwerenności jest tym, co powstrzymuje przed tym, 
aby kogoś manipulować, kontrolować lub wykorzystać. 
Temat odnośnie przywództwa do rozwinięcia przy innej okazji.

\subsubsection{Całkowicie akceptuj to takim, jakim jest. Nie usiłuj tego zmieniać wbrew niemu.}
Przyjmij siebie samego takim, jakim jesteś.
Akceptuj swoje wady i niedoskonałości, to jest piękne, to sprawia że jesteśmy ludźmi.
\emph{Do rozwinięcia.}

\subsubsection{Nie potrzebuj niczego od tego innego.}
Nie próbuj wykorzystać nikogo jako narzędzie do spełnienia swoich samolubnych potrzeb.
Potrzeba tworzy przywiązanie.
Jeżeli jako dorośli ludzie polegamy w naszym przetrwaniu na kimś, 
krzywdzimy w ten sposób nas obu.
\emph{Do rozwinięcia.}

\subsubsection{Respektuj tego światopogląd i chciej go zrozumieć.}
To jest co sprawia, że kochanie innych ludzi, czy nawet zwierząt, 
jest inne od kochania rzeczy, pasji, itd..
Każdy człowiek ma swój światopogląd i postawę, którą się odznacza, 
a to jest pewne, że pojawią się różnice między tym, a twoim światopoglądem i postawą. 
Jeżeli w każdym przypadku nie potrafisz zatrzeć różnic między wami w 
zniekształcaniu twojego postrzegania i kochania kogoś, 
to będziesz niezdolny do kochania innych ludzi. 
Do tej zasady, musisz być na prawdę otwarty na inne światopoglądy, różne od twojego. 
To niemożliwe głęboko obdarzyć kogoś Miłością, gdy jest się hermetycznie zamkniętym w sobie, 
ponieważ nie jest się wystarczająco elastycznym w epistemologii, 
aby móc czyiś światopogląd przyjąć i zrozumieć. 
Jeżeli na prawdę kogoś kochasz, to chcesz go zrozumieć.

Zrozumienie czyjegoś światopoglądu jest trudnym zadaniem. 
To bardzo powszechne, żeby uniknąć tego problemu 
i przeskoczyć konieczność zrozumienia kogoś, gdy próbujemy kochać. 
Znowu przykładem mogą być rodzice. 
Niewielu na prawdę próbuje poznać swoje zwłaszcza nastoletnie dzieci. 
Niewielu próbuje poznać ich potrzeby, zrozumieć ich styl oraz potem ich w tym wspierać. 
Zrozumienie czyjegoś światopoglądu może generować konflikty, 
w końcu próbujesz pojąć różnice między waszymi postawami, 
które fundamentalnie budują naszą tożsamość. 
Czasami sie okazuje, że jest się całkowicie niekompatybilnym z drugą osobą, 
dla niektórych to jest bolesne do przełknięcią i mimo tego próbują funkcjonować ze sobą, 
krzywdząc siebie samych w tym procesie np. tworząc dysfunkcyjny związek z taką osobą. 
Czasami zalewamy kogoś własnym światopoglądem, 
nie słuchamy go, ale przytłaczamy własnymi myślami, 
aby ich myśli wyciszyć z konwersacji, w tej walce paradygmatów 
-- jest to jakaś forma dominacji, która gryzie sie z budowaniem głębszych relacji z ludźmi. 
Czasami sobie nie poradzimy ze zrozumieniem czyjegoś światopoglądu, 
gdy jest tak całkowicie inny i obcy od tego, co znamy, 
to też często jest podstawą do demonizacji i odrzucania czyjegoś światopoglądu, 
bo prawdziwe zrozumienie jest trudne. 
Jesteśmy ludźmi. 
Mamy różne potrzeby, różne doświadczenia, różne budowy ciał, umysłu i sposoby przetrwarzania informacji. 
Mamy inne kultury, inne środowiska w jakich przyszło nam żyć. 
Te różnice nie mogą sprawić, że zapomnimy o tym, że jesteśmy tym samym, ludźmi. 

\subsubsection{Słuchaj, co próbuje się Tobie mówić.}
Taka prosta rzecz jak słuchanie często wystarczy w zdrowej komunikacji z ludźmi. 
Kochać oznacza stanąć przed kimś i go wysłuchać, nie przerywać mu i być cierpliwym. 
Więc też, musimy poświęcić temu komuś czas, uwagę i uważność. 
\emph{Do rozwinięcia.}

\subsubsection{Troszcz się o Tego interesy oraz współdziel z nim swoje własne.}
Realistycznie, troszczenie się o czyjeś interesy i współdzielenie swoich interesów z nimi 
jest podstawą do budowania i utrzymywania głębokich połączeń z ludźmi. 
Bo dlaczego próbujesz budować z kimś taką relację, 
jeśli nie macie nic wspólnego ze sobą, jeśli nic was nie łączy? 
W takich przypadkach nie ma potrzeby budowania jakiegokolwiek związku, 
a jedynie działa się w ramach poczucia obowiązku lub w akcie desperacji. 
Z poczuciem obowiązku budowania takich relacji często sie spotka w rodzinnych relacjach. 
Mamy przekonanie, że skoro z kimś jesteśmy powiązani rodzinnie,
to mamy obowiązek ich kochać, nawet jeśli nie ma się nic wspólnego z tą osobą, 
albo gdy ta osoba jest chujem w stosunku do nas. 
Więc co robimy to próbujemy udawać miłość i zainteresowanie. 
Ale ta rzecz z miłością, prawdziwej \emph{Miłości} nie można udawać 
-- nią się jest -- i o tym fałszu oboje będziecie podświadomie wiedzieć.
Czasami różnice w wartościach i zainteresowaniach ludzi jest zbyt duża, 
aby znaleźć jakąś przestrzeń do wspólnego obcowania. 
Bo widzisz, działanie z poczucia obowiązku lub w akcie desperacji, 
z potrzeby budowania z kimś takich głębszych relacji, jest niewystarczająca w Miłości. 
Taka niezdrowa postawa jest fundamentem do budowania nienawiści i zniechęcenia między wami. 

\subsubsection{Praw werbalne aprobaty, komplementy i pochwały.}
\emph{Do rozwinięcia.}

\subsubsection{Bądź hojny, życzliwy i dobry.}
Oferuj pomoc i wsparcie, gdy to potrzebne.
\emph{Do rozwinięcia.}

\subsubsection{Broń tego, bądź temu lojalny.}
Jesteś w stanie powiedzieć, gdy ktoś jest Tobie lojalny.
Albo gdy ktoś nie jest -- jest Twoim wrogiem, knuje przeciwko Tobie za Twoimi plecami, 
darzy Cię skrytą lub otwartą nienawiścią.
To można wyczuć swoją intuicją.
Zdajesz sobie sprawę, że nie można zaufać ludziom, 
którzy decydują się Ciebie zaniedbać lub stają przeciwko tobie.
Nie można na nich polegać, bo nie są Tobie lojalni.
Obcowanie z ludźmi, których nie darzysz zaufaniem, buduje w Tobie stres i wyizolowanie.
Nie pozwól na siebie wpływać ludziom, którzy są przeciwko Tobie i Twoim interesom.
Bądź z ludźmi, którzy są dla Ciebie i darzą Cię Miłością.
Lojalność i zaufanie raz złamane praktycznie niemożliwym jest odbudować.
Dlatego pamiętaj o tym, aby nigdy nie pozwolić sobie tego złamać.
Broń ludzi, których kochasz.
Bądź lojalny dla nich i stań po ich stronie.

\subsubsection{Buduj oraz utrzymuj zaufanie i poczucie bezpieczeństwa.}
Prawdziwość buduje zaufanie. 
Kłamstwo, manipulacja, projekcja, itd. 
tworzą podstawy do nieufności, nienawiści, strachu.
Ludzie w strachu i nieufności, nie czują się bezpiecznie. 
Jeżeli nie jesteś lojalny do kogoś, albo jesteś nieprawdziwy, 
wtedy wiadomo że nie można na tobie polegać. 
To ważne, aby dbać o poczucie zaufania i bezpieczeństwa zawsze, 
od samego początku, bo raz zaburzone zaufanie ciężko odbudować, 
praktycznie jest to niemożliwym w większości przypadkach. 
Nie buduj zaufania, aby potem kogoś wykorzystać i nim manipulować, to nie jest Miłość. 
Nigdy nie wyzyskuje się raz zbudowanego zaufania. 
Zaufanie buduje się właśnie po to, 
aby budować poczucie bezpieczeństwa i je pielęgnować. 
 
\subsubsection{Buduj poczucie wspólnoty, związku, przyjaźni i współpracy.}
To wynika z potrzeby budowania połączeń w praktyce \emph{Miłości}. 
Nic tak nie pogłębia tych więzi, jak kolaboracja, 
wspólna praca nad czymś, wspólne doświadczanie czegoś, 
dzielenie się z kimś swoimi myślami, ideami, zasobami.
\emph{Do rozwinięcia.}

\subsubsection{Uchroń to przed nadmierną traumą, strachem i cierpieniem, które może to wyniszczyć i pokaleczyć.}
To jest tak na prawdę rola, jaką pełnią rodzice, 
że próbują uchronić swoje dziecko przed traumą i cierpieniem.
Ale nie można dopuścić, by dziecko nigdy nie doświadczyło strachu ani cierpienia.
Nadopiekuńczość skrzywdzi je tak samo, a nawet bardziej, jak nadmierne cierpienie, 
gdy to dziecko wejdzie w dorosłość i zacznie być odpowiedzialne za siebie.
Nie uciekaj od cierpienia i chorobliwości, bo od tego nie uciekniesz.
To ważne, aby nauczać jak cierpieć świadomie 
oraz by nie trzymać tego cierpienia w sobie i pozwolić mu odejść.
Wszyscy cierpimy wystarczająco, niepotrzebnie.
Nakładając sobie więcej niepotrzebnego stresu i cierpienia na siebie dążymy do samodestrukcji.
Może to nie być oczywiste, ale takie chorobliwe drobnostki 
budujące się na nas mogą nas poważnie skrzywdzić, a nawet doprowadzić do samobójstwa.

W kontekście nadopiekuńczości -- jeżeli coś kochasz, 
nie chcesz aby to było uzależnione od Ciebie do końca swojego życia.
Chcesz aby stało się silniejsze i w końcu niezależne od Ciebie.
A wręcz chcesz, aby stało się silniejsze od Ciebie i rozwinęło się bardziej niż Ty, dzięki Tobie.
W przypadku Miłości do Ja mamy podobną sytuację, 
tylko porównujemy siebie tu i teraz do swojego Ja z wczoraj.
Chcemy być silniejsi, niż byliśmy wcześniej. 
Chcemy się stale rozwijać i rosnąć, przekraczać własne limity.
To rola Miłości w tym, aby rosnąć w zdrowy dla nas samych sposób.

W kontekście zasad zdrowej komunikacji 
-- ludzie są i będą niezdrowi, nieświadomi, chorobliwi, egoistyczni.
Jeżeli nie potrafisz w zdrowy sposób podejść do niezdrowego człowieka, 
to znaczy nie jesteś gotowy kochać tego, co Ciebie nienawidzi, 
kogoś tak nienawistnego i nieświadomego, 
to Twoja odpowiedzialność, aby siebie samego przed tym uchronić.
Nie angażuj się w niepotrzebne konflikty.
Unikaj relacji z ludźmi, którzy wciągają Cię w swoją chorobliwość.
Uchroń siebie przed niepotrzebnym cierpieniem.

\subsubsection{Bądź wdzięczny za to, co masz.}
Zamiast skupiać się na tym, co Ci brakuje.
\emph{Do rozwinięcia.}

\subsubsection{Bądź otwarty na naukę i nowe doświadczenia.}
Zwłaszcza jeśli wymagają od Ciebie wyjścia poza Twoją strefę komfortu.
Odkrywaj siebie samego. Poznaj swoje pasje i wizję, której możesz się poświęcić.
\emph{Do rozwinięcia.}

\subsubsection{Bądź dla tego kogoś, gdy czuje się gorzej, żałośnie, słabo, beznadziejnie, zdesperowany, biednie.}
Nie bez powodu istnieje powiedzenie, że 
prawdziwych przyjaciół poznaje się w biedzie. 
Łatwo jest być dla kogoś, gdy jest silnym, szczęśliwym, 
w jej czasie bycia najlepszą wersją siebie, 
gdy twój partner zdobędzie znaczną podwyżke w pracy, albo wypali wasz biznes. 
Czujemy się przy takich ludziach komfortowo, 
to nie jest prawdziwy sprawdzian Twojej Miłości. 
Lecz co jeśli zdarzy im sie chwila słabości? 
Popełnią błąd, wydarzy się w ich życiu coś, co ich chwilowo odeśle w żałobe? 
Gdy przestanie im się powodzić i potrzebują pomocy przetrwać? 
Gdy potrzebują być wysłuchanym? 
Gdy dostaną diagnozę o śmiertelnej chorobie od lekarza? 
Gdy stracą pracę i nie mają za co chleba kupić? 
Gdy chcą popełnić samobójstwo? 
Prawdziwym sprawdzianem w Miłości jest kochać kogoś, kto uderzył w dno. 
Kto jest najsłabszą wersją siebie kiedykolwiek. 
Być dla kogoś w tym czasie, to prawdziwy sprawdzian. 
Bo jeśli na prawdę kogoś kochasz, 
chcesz być dla tej osoby w każdej chwili, nawet w tej najgorszej. 
Bo pomyśl, co to świadczy o tobie, 
gdy dystansujesz się od kogoś i odłączasz od tej osoby w momencie, gdy dotknie dna. 
Chciałbyś być kimś takim dla siebie samego? 

\subsubsection{Pomagaj temu nie czuć się samotnie.}
Bardzo łatwo jest nam ludziom wpaść w samotność, zwłaszcza w dzisiejszych czasach. 
Właściwie ciężko nie znaleźć kogoś, kto ma depresję lub czuje się samotny. 
Najgorsze jednak są przypadki poczucia samotności wśród ludzi, 
bo tutaj próbując pielęgnować własną potrzebę udziału we wspólnocie, 
często trafiamy do środowisk które nas nie doceniają 
i czujemy się samotnie, chociaż otoczeni ludźmi. 
Ale należałoby się tutaj zastanowić, na ile ta samotność to prawda, a na ile nasza własna projekcja.
Często sami wybieramy tą samotność, 
nie potrafiąc docenić to co ktoś robi dla nas i oczekując więcej. 
Problem może leżeć po obu stronach. 
Samotność to zdecydowanie temat, który warto poruszyć głębiej przy innej okazji. 

\subsubsection{Zatwierdzaj, współdziel i dbaj o uczucia tego. Nie ignoruj ani nie zaniedbuj uczuć.}
To jest co pozwala na zbudowanie emocjonalnej więzi z czymś lub kimś. 
Musisz potrafić być empatyczny. 
Magią jest tak prosta rzecz, jak zapytanie kogoś „Jak się czujesz?”. 
Po prostu. Jak się czujesz? 
Czego potrzebujesz? 
Pozwól, że Ja Ciebie wysłucham. 

Więc pytaj ludzi jak się czują, słuchaj ich i zatwierdzaj te uczucia. 
Zatwierdzaj, czyli nie przecz im 
-- \emph{a bo ty jesteś dziwna, w ogóle szalona, to niewłaściwe czuć się w ten sposób.} 
Nawet jeśli sam czujesz nieracjonalność, brak sensu w czyichś uczuciach, musisz je zaakceptować. 
I to uczucie w sobie również. 
To że czujesz się inaczej od kogoś innego, nie neguje ich uczuć w tej samej sprawie. 
Właśnie to, nasze uczucia i odmienności, nieperfekcyjności związane z tym, 
jak kształtują nas nasze uczucia, sprawia że jesteśmy ludźmi. 
I to jest piękne. 
Ważną kwestią tego jest również współdzielenie uczuć z innymi. 
To rola empatii, ale też wynika z bliskości i tego, 
że traktujemy kogoś jako rozszerzenie nas samych. 
Ich cierpienie to nasze cierpienie.
Ich śmiech to nasz śmiech.

\subsubsection{Troszcz się o tego cierpienie, dla cierpienia samego w sobie.}
Najczęstrzym błędem tutaj jest troszczenie się o cierpienie kogoś nam bliskiego, 
aby pomóc im je szybciutko skończyć 
-- w końcu ich cierpienie to nasze cierpienie, 
jak sobie przed chwilą powiedzieliśmy. 
Ale to nie jest w istocie troszczeniem się o cierpienie dla jego samego w sobie, 
tylko dlatego że to nieszczęście u kogoś nas denerwuje, 
gdy ta osoba obrzuca nas swoimi łzami i wyciąga nas z naszej strefy komfortu, 
więc chcemy się jakoś szybko tym zająć w ich sprawie. 
Brakuje tutaj pola do tego, aby cierpieć z kimś innym 
i żeby obchodziło nas czyjeś cierpienie z samego faktu tego cierpienia. 
Można przytoczyć z czatu: Cierpienie dotyka nas chorobliwością i nieświadomością. 
Cierpienie jest narzędziem zmiany, które subtelnie naucza nas jak kochać i jak żyć świadomie. 
Spróbować kochać w chwili największej słabości, 
gdy żyje się w nienawiści i nieświadomości, 
jest najtrudniejszym, co można tu i teraz zrobić. 

\subsubsection{Utrzymuj pokój. Nie wprowadzaj niepotrzebnych konfliktów.}
\emph{Do rozwinięcia.}

\subsubsection{Bądź z tym na równych warunkach.}
Zwłaszcza w przypadku, gdy ktoś jest mniej rozwinięty emocjonalnie czy intelektualnie od nas, 
to nasze zadanie dostosować naszą postawę do kogoś tak, 
aby mógł nas zrozumieć, aby zbudować połączenie w naszej komunikacji. 
Jeżeli rozmawiam z człowiekiem uzależnionym od narkotyków, 
w kontekście udzielenia mu pomocy, 
to ważne aby dostosować swoją komunikację do jego poziomu. 
Bo byłbym totalnym idiotą, jeśli w takiej sytuacji zacząłbym gadać o Bogu, 
nieskończonej Miłości, metafizyczności Wszechświata 
czy czymkolwiek co tego człowieka nie obchodzi 
i nie daje realnych podstaw do wniesienia pomocy w jego problemie. 
Jeżeli próbujemy nauczyć czegoś kogoś, to ważne żeby pozwolić mu nas zrozumieć. 
Jeżeli próbujemy nauczyć kogoś czegoś, na co on nie jest gotowy, to ważne żeby go do tego przygotować. 
To jest właśnie bardzo frustrujące, 
próbować przedstawić komuś idee, które na jego poziomie będą niezrozumiane 
-- ale to nasz problem, bo co powinniśmy zrobić to dostosować się tak, aby umożliwić komunikację. 
Być z tym na równych warunkach nie mówi o tym, aby okaleczać siebie samych, 
by człowiek słaby i niezdrowy mógł obcować z nami na jego warunkach. 
Nie chodzi też o to, aby zaniżać się do czyjegoś poziomu, 
właśnie w ten niezdrowy dla nas bądź krzywdzący sposób. 
To ważne, żeby dbać i kochać siebie samego. 
Miłość pomaga nam się rozwinąć i uzdrowić, chodzi właśnie o to, 
aby każdemu pozwolić na to, by być kochanym, 
niezależnie od tego na jakim poziomie znalazł się w tej chwili. 
Nie wiem czy tym wywodem zdołałem to przedstawić w takiej postaci, 
żeby ta myśl była oczywista, najwyżej będę do tego wracać i redagować na nowo.

\subsubsection{Okazuj temu zrozumienie i wybaczenie.}
Okazuj zrozumienie zwłaszcza, gdy ktoś popełnia błędy. 
Wybaczaj ludziom błędy, które popełnili. 
Nie trzymaj w sobie i nie żyw do tego urazy.
Nie demonizuj ani nie moralizuj.
Jako ludzie to naturalne, że popełniamy błędy, 
że nie wiemy czegoś wystarczająco, 
że dochodzi do nieporozumień i konfliktów między nami, 
że nie zrobimy czegoś, albo zrobimy coś inaczej, niż powinniśmy. 
Kwestia wybaczenia jest tu bardzo ważna w utrzymywaniu zdrowych relacji z ludźmi. 
Jeżeli nie potrafilibyśmy nikomu wybaczać, 
zwłaszcza gdy wchodzimy na personalny poziom czegoś, kto coś zrobił przeciwko nam, 
tracimy zaufanie w ludziach, widzimy w nich wrogów, 
zaczynamy w innych dostrzegać ryzyko zostania skrzywdzonym i izolujemy się od nich. 
To jest niezdrowe i wpycha nas w samotność, albo w radykalne i skrajne poglądy i ideologie.
Znamy wiele przypadków, jak się to kończy; 
szaleńców, którzy dokonują masowych mordów, 
dlatego że nie potrafili wybaczyć krzywdy, której ktoś im dokonał, 
więc mszczą się na niewinnych ludziach. 
Ludzie, którzy siebie nie kochają, nie potrafią sobie wybaczyć popełnionych błędów. 
Dręczą ich myśli, że postąpili inaczej niż ich idealne Ja by zrobiło. 
Nie akceptują tego, że przez ich słabości odeszła z ich życia osoba, 
z którą wiązali swoją przyszłość. 
Albo przywołują tą jedną głupią rzecz, 
którą powiedzieli w konwersacji z nieznajomym 7 lat temu. 
Musimy nauczyć się wybaczać, zwłaszcza sobie samym. 
Dlatego to ważne, aby włożyć wysiłek w zrozumienie siebie samych 
-- to pozwala nam zrozumieć innych i to jest podstawą do wybaczenia. 

\subsubsection{Uświadamiaj to o tym, że wszystko jest w porządku.}
\emph{Do rozwinięcia.}

\subsubsection{Bądź cierpliwy i zaczekaj.}
To ważne być cierpliwym, aby potrafić się poświęcić i w skupieniu słuchać lub obserwować. 
Cierpliwość jest ważna, gdy próbujemy zrobić coś trudnego. 
Nauczenie się matematyki może być czymś takim. 
Już wielokrotnie mówione było, że Miłość jest trudna, 
więc tutaj przede wszystkim trzeba być cierpliwym. 
Bycie niecierpliwym jest małą czerwoną flagą, 
która mówi, że w istocie nie kochamy tego czegoś lub kogoś, 
że nie mamy czasu dla tego, 
że wchodzi nam pod skórę i nas denerwuje. 
Ogólnie temat cierpliwości można rozwinąć, przy okazji tematu jak np. 
„Dlaczego rozwinięcie wartościowych rzeczy i umiejętności w życiu wymaga czasu?” 

\subsubsection{Nie uciekaj, nie usiłuj, ani nie bądź bierny.}
\emph{Do rozwinięcia.}

\subsubsection{Dostrzegaj Miłość w tym, nawet jeśli ono samo w sobie tej Miłości nie dostrzega.}
Sztuką jest dostrzec Miłość w kimś, 
kto jest pochłonięty w swoim egoiźmie, nienawiści, chorobliwości. 
Z pewną ciekawą rzeczą się spotkałem robiąc research: 
\emph{jedną z najwyższych form Miłości jest dostrzec w diable Boga i to też sposób, by zmienić diabła w anioła.} 
Czyli, aby widzieć w kimś Miłość i dobro, nawet jeśli sam tego nie widzi. 
Jeżeli ktoś jest niemiły i okrutny do siebie, 
jesteś w stanie go w tym przejrzeć i dostrzec dobro oraz prawdziwe piękno. 
Jeżeli ktoś jest nienawistny do siebie, nie zrównoważa Cię to, 
tylko jesteś w stanie w nich spojrzeć i zobaczyć Miłość, z której powstał. 
To jest coś bardzo zaawansowanego i wymagałoby prawdziwego poświęcenia w praktyce Miłości, 
aby być na codzień w stanie dzielić taką postawę z ludźmi, 
którzy są wcieleniem zła i chorobliwości. To temat do rozwinięcia. 

\subsubsection{Rozpoznaj w tym niepowtarzalność i unikatowość.}
Wszystko jest w jakiś sposób niepowtarzalne i unikatowe. 
Wszystko ma swoje miejsce w przestrzeni i czasie, 
każdy człowiek ma unikatowe geny i doświadczenia, które kształtowały go przez całe życie. 
Częścią kochania kogoś jest rozpoznanie tej unikatowości, 
jego indywidualnej ekspresji siebie, 
zdanie sobie sprawę z tego że osoba czy rzecz, którą masz przed sobą, jest niepowtarzalna. 
Nigdy nie znajdziesz kogoś identycznego. 
Nie próbuj pozbawiać ludzi ich unikatowości i próbować zrobić tak, 
by byli dokładnie tym samym, jak stereotyp, albo żeby pasowali do twojej fantazji. 
Unikatowość przejawia się w każdej formie, 
zwłaszcza łatwo ją dostrzec w różnicach, jakie sie tworzą między wami. 
To ważne, aby traktować ludzi względem tego, czym są na prawdę, 
a nie względem tego co zakładamy czym są, 
skoro są podobni do czegoś z czym już się spotkaliśmy. 
Tylko, czy na pewno? 
Bo wrzucając ludzi do jednego worka pozbawiamy siebie szansy na to, 
żeby ich poznać na prawdę, nie przez pryzmat czegoś, co zniekształca naszą wizję o nich. 
Jest też ciekawy \href{https://youtu.be/NrNpf1ZP8i4}{film Kacpra Pitały}, czy istnieją dwie identyczne rzeczy.

Też jedna rzecz odnośnie randkowania. 
Jako facet, jeśli na prawdę chcesz zdobyć serce dziewczyny, 
\emph{zakomunikuj do niej jej unikatowość.}
Jeżeli dajesz jej komplement, 
zakomplementuj ją w kontekście tego, co sprawia że jest unikatowa 
-- nie w kontekście tego, co sprawia że jest taka jak inne dziewczyny. 
Każda to doceni. 
Właściwie, to Ciebie też postawi w zupełnie innym świetle w jej oczach 
od całej reszty facetów, z którymi pewnie miała styczność. 
To ogólnie nie tylko prawda przy komplementowaniu dziewczyn jako faceci. 
Nawet jeśli jesteś szefem i komplementujesz swoich pracowników, 
gdy jesteś liderem i dajesz komplementy ludziom, z którymi przychodzi Ci współpracować, 
nic nie będzie tak potężne jak zwrócić uwagę na to, 
co sprawia że są niepowtarzalni, co tworzy ich unikatowość. 

\subsubsection{Dostrzegaj w tym wewnętrzne, nieskończone piękno.}
Im bardziej świadomy i skupiony jesteś, 
tym bardziej dostrzegasz piękno w rzeczach i ludziach wokół Ciebie. 
Ale musisz zbudować z tym połączenie, 
aby móc na prawdę docenić to piękno i im bardziej coś kochasz, 
tym bardziej jesteś chętny by to zrobić. 
Ludzie nie dostrzegają piękna, 
przez swoją stronniczość, samolubność, niezdrowe potrzeby, zapatrzenie w siebie samego, 
czy przez to że żyjemy w fantazji. 
Pewien drastyczny przykład z którym się spotkałem robiąc research, 
jest równe piękno w zaobserwowaniu szczątków przejechanego kota na poboczu, 
co w dziele sztuki jak np. obraz. 
I tak jako ludzie, martwe zwierze zabite przez człowieka 
budzi w nas odrazę, zwątpienie i może nienawiść do tego, kto przejechał biedne zwierzę, 
z drugiej strony odwiedzając muzeum wpatrujemy się w obraz przedstawiający tą samą scenę;
wpatrując się w niego godzinami i kontemplując nad jego pięknem. 
To subtelny sposób w jaki jako ludzie pokazujemy naszą stronniczość. 
Rozwijając przykład, z drugiej strony możemy mieć kogoś, kto odwiedzając muzeum, 
spojrzy się na ten sam obraz, powie pod nosem "mhmm, ciekawe" i pójdzie dalej. 
Nie interesuje go to w żaden sposób. 
Nie dostrzega w tym żadnego piękna. 
Tak płytkie połączenie zostało stworzone między tej osoby świadomością, 
a tym obrazem z którym przyszło mu obcować. 
Tymczasem artystę, czy konesera sztuki, 
wzruszy każde pojedyńcze maźnięcie pędzlem, dobór kolorów, faktura na płótnie, 
samą scenę którą ten obraz przedstawia, do stopnia 
że może go nawet doprowadzić do płaczu, gdy widzi to piękno przed sobą. 

\subsubsection{Dotrzymuj swoich obietnic.}
\emph{Do rozwinięcia.}

\subsubsection{Mów prawdę i bądź prawdziwy.}
Nie kłam. 
Nie udawaj nikogo, kim nie jesteś. 
Nie dotyczy to tego, aby nie próbować zmieniać się w kogoś innego.
Właściwie rzecz biorąc, jeśli oczekujesz personalnej zmiany na jakimś poziomie, 
musisz być tym, w co próbujesz się zmienić, tu i teraz.
Ludzie często nie zdają sobie sprawę z tego jak bardzo istotne jest to, aby być prawdziwym. 
Nie dostrzegają połączenia między Prawdą, Świadomością i Miłością, bo to już nie jest oczywiste. 
Nie zbudujesz Miłości na fundamencie kłamstwa. 
Prawda, Miłość i Świadomość zawsze były synonimami, 
bo praktykować jedno oznacza praktykować je wszystkie. 
Nie są jakimś atrybutem drugiego, są absolutem, są jednym. 
Kwestie Prawdy i prawdziwości również można rozwinąć przy innej okazji. 

\subsubsection{Dostrzegaj w tym tego prawdziwość.}
Jeżeli widzisz coś, powinnieneś widzieć to takim, jakie jest na prawdę. 
Nie w żaden zniekształcony sposób
lub w sposób w jaki chciałbyś żeby było w przyszłości, 
lub czym było kiedyś. 
To bezpośrednio łączy Prawdę z Miłością. 
Bo Miłość musi być rozpoznaniem i uznaniem prawdziwości istnienia. 
Jeżeli widzisz obiekt, musisz go kochać dokładnie takim, jakie jest, 
inaczej go nie kochasz. 
Bo jeżeli nie lubisz tego jakie jest, 
jeżeli nie akceptujesz tego i chciałbyś to zmienić, 
to znaczy tak na prawdę nie widzisz co tam jest. 
Nie dostrzegasz prawdziwości tego, więc zwracasz się do fantazji. 
Ale to nie jest tym, co jest przed tobą.

\subsubsection{Całkowicie akceptuj i nie bądź niezrównoważonym przez tego egoizm, ani nie usiłuj tego postawy zmieniać.}
W innych słowach, jeśli kochasz coś, 
jakiś skończony skrawek świadomości, jaki masz przed sobą, 
musisz zdać sobie sprawę z jednej ważnej rzeczy i ją zaakceptować
-- ten ktoś będzie egoistyczny, będzie dbało o swoje przetrwanie.
I nie możesz próbować tego zmieniać. 
Jeżeli ludzie z którymi mamy doczynienia są samolubni, 
będziemy próbować to zmienić, bo nie lubimy tego egoizmu. 
Taka jest nasza egoistyczna natura.

Taki przykład, napisałem sobie tą liste o tym co oznacza kochać, prawda? 
Powiedzmy że pójdę teraz do rodziny, przedstawię im ją, 
złożymy sobie obietnice że będziemy ją praktykować razem, 
ale wtedy ktoś z rodziny zachowa się jak chuj albo egoistycznie w stosunku do mnie, 
w sprawie jakiejkolwiek, arbitralnej,
rodzina trzymałaby jego stronę, a ja byłbym zły na nich i chciałbym ich zmienić. 
Więc nie kocham więcej bo byli samolubnymi chujami. 
ALE to byłby właśnie sprawdzian mojej Miłości. 
Bo znowu poruszając ten temat, 
sprawdzianem Miłości nie są sytuacje, kiedy ktoś nas kocha i jest bezinteresowny do nas. 
Prawdziwy sprawdzian Miłości, to kochać tych, co są chujami do nas, co darzą nas nienawiścią. 
Jak bardzo jesteś w stanie kochać ludzi bardzo niezdrowych, nienawistnych, egoistycznych i nieświadomych? 
To jest prawda o nich, ich nieświadomość jest prawdziwa, można ją zaobserwować. 
Dla Ciebie by ich kochać to kochać ich jakimi są na prawdę. 
Najwyższa forma bezinteresowności, 
to umiejętność by nie być niezrównoważonym przez egoizm i całkowita akceptacja egoizmu u kogoś. 
To również fundamentalna zasada, na której bazują zasady zdrowej komunikacji. 
Z natury jesteśmy egoistami. 
Miłość polega na tym, aby stawać się bezinteresownym.

\subsubsection{Pozwól odejść Temu, co kochasz. Nie usiłuj Tego zatrzymać dla siebie wbrew Temu.}
To temat dotyczący akceptacji zmiany. 
Można też odwołać się do wielu przytoczonych przykładów, 
które dotyczyły życiem w fantazji i na jej podstawie obcowanie z innymi. 
Więc też to, że nie widzimy prawdziwości świata wokół nas, 
nie akceptujemy tego, więc próbujemy go zmienić tak, aby pasował naszej wizji dla świata. 
Rozstania są bolesne. 
Trudno jest zaakceptować, że osoba którą kochaliśmy, 
z którą budowaliśmy wspólną przyszłość, 
w jednej chwili znika z naszego życia, 
zrywa całe połączenie jakie między sobą pielęgnowaliśmy. 
Próby zatrzymywania kogokolwiek wbrew temu 
to nie tylko niezdrowy sposób na budowanie nienawiści do nas w drugiej osobie, 
to także sposób na straszliwe zniewolenie samego siebie. 
Odchodząc od tematu związków, jest też koncept „Więźnia Umysłu”, który właśnie dotyczy tego zniewolenia. 
Nie pozwalamy naszym niezdrowym myślom i uczuciom odejść, 
trzymamy je w sobie, a one budują się na nas i tworzą chorobliwość. 
Później kurczowo trzymamy się tej chorobliwości i nie pozwalamy jej odejść. 
Osobiście doszedłem do takiego wniosku, przy okazji pisania do siebie w dzienniku. 
Sposobem na uzdrowienie siebie samego, zwłaszcza w tym kontekście, to obdarzyć siebie Miłością. 
Żeby pozwolić odejść chorobliwości w Tobie, musisz ją kochać. 
W ten sposób ją rozwiązujesz, i możesz to zrobić chociażby poprzez medytację. 
Ostatnim co chciałem przytoczyć, to temat śmierci.
Ale do tego chyba wrócę przy innej okazji. 

\ornamentbreak

\vspace*{-1.5em}
\begin{multicols}{2}


\subsection{Brak Miłości}
\label{sztuka/milosc/brak-milosci}

Brakiem/złamaniem Miłości są:
przemoc, znęcanie się, nękanie, nienawiść, złość, osąd, krytyka,
strach, kłamstwo, wyzysk, zdrada, oszustwo, kradzież,
demonizacja, moralizacja, obarczanie winą, nieakceptacja, 
brak szacunku, kpina, złamanie obietnic, projekcja, gaslighting, manipulacja,
obojętność wobec cierpienia, karanie - zwłaszcza za popełnione błędy,
zemsta, zamknięcie się w sobie, egoizm, narcyzm, samolubność,
nieświadomość, usiłowanie zmieniania kogoś.

\emph{Do rozwinięcia.}

\columnbreak


\subsection{Jak zakochać się w życiu?}
\label{sztuka/milosc/jak-zakochac-sie-w-zyciu}
\emph{Do rozwinięcia.}


\subsection{Miłość do siebie}
\label{sztuka/milosc/milosc-do-siebie}
\emph{Do rozwinięcia.}


\end{multicols}


\end{document}
